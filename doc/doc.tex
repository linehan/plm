%%%%%%%%%%%%%%%%%%%%%%%%%%%%%%%%%%%%%%%%%%%%%%%%%%%%%%%%%%%%%%%%%%%%%%%%%%%%%%%%
% LaTeX template
%%%%%%%%%%%%%%%%%%%%%%%%%%%%%%%%%%%%%%%%%%%%%%%%%%%%%%%%%%%%%%%%%%%%%%%%%%%%%%%%

\documentclass[10pt, fleqn]{article}

%%%%%%%%%%%%%%%%%%%%%%%%%%%%%%%%%%%%%%%%%%%%%%%%%%%%%%%%%%%%%%%%%%%%%%%%%%%%%%%%
% PACKAGES 
%%%%%%%%%%%%%%%%%%%%%%%%%%%%%%%%%%%%%%%%%%%%%%%%%%%%%%%%%%%%%%%%%%%%%%%%%%%%%%%%

%
% T1: 8-bit encoding to allow codepoints for 
%     fonts with up to 256 glyphs
% 
%     Without T1, the following problems arise:    
%         -Cannot use many newer fonts
%         -Words containing accented characters 
%          cannot be hyphenated
%         -You cannot copy-and-paste words with
%          accented characters from the output 
%          (DVI/PS/PDF)
%         -Characters like '|', '<', and '>' 
%          can render in unexpected ways. 
%
\usepackage[T1]{fontenc} 

%
% Allows specification of arbitrary font size,
% even sizes not listed in the .fd file.
%
%        \fontsize{...}{...}\selectfont
%
% The nearest size is selected, then scaled to
% the requested size.
\usepackage{anyfontsize}

%
% Math script text will use the 'Euler' font
% rather than the default.
%
%        \mathscr{...} 
\usepackage[mathscr]{euscript} 

%
% AMS Packages
%
\usepackage{amsmath}           % AMS package providing \equation, \align, etc. 
\usepackage{amssymb}           % AMS package providing math symbols 
\usepackage{amsthm}            % AMS package providing theorem environments

%
% Graphics
%
\usepackage{graphicx}          % Allows different kinds of image to be embedded 
\usepackage{wrapfig}           % Allows text wrap around embedded figures 
\usepackage{hyperref}          % Allows creation of hyperlinks

% 
% Page geometry
%
\usepackage{fullpage}          % Uncomment to enable the full page to be used
\usepackage{changepage}        % Allows mod. of page geometry (\adjustwidth)
\usepackage{fancyhdr}          % Allows definition of page headers
\usepackage{marginnote}        % Allows margin notes

%
% Columns and margins
%
\usepackage{multicol}          % Provides '\begin{multicols}{n}' 
\usepackage{parcolumns}        % Provides '\lcolumn', '\rcolumn' environments 
\usepackage{enumitem}          % Provides '\enumerate' environment enhancements
\usepackage{marginnote}        % Provides '\marginnote' environment 

%
% Special symbols
%
\usepackage{turnstile}         % Turnstile symbols, like |-, |=
\usepackage{cancel}            % Lets you negate anything with a slash 
\usepackage{mathtools}         % \xrightarrow{hello} for hello
                               %                         ---->

% 
% Boxes and structure
%
\usepackage{framed}            % \framed{...}           Draw a framed box
                               % \oframed{...}          Open at page breaks
                               % \shaded{...}           Shade 
                               % \shaded*{...}          Shade to margin
                               % \snugshade{...}        Shade fit
                               % \snugshade*{...}       Shade fit to margin
                               % \leftbar{...}          Bar on left side
                               % \titled-frame{...}     Frame with title bar

\usepackage{fancybox}          % \doublebox{...} \shadowbox{...} \ovalbox{...}
\usepackage{color}             % \color[rgb]{r,g,b}{...}

\usepackage{minted}

%%%%%%%%%%%%%%%%%%%%%%%%%%%%%%%%%%%%%%%%%%%%%%%%%%%%%%%%%%%%%%%%%%%%%%%%%%%%%%%%
% ENVIRONMENTS 
%%%%%%%%%%%%%%%%%%%%%%%%%%%%%%%%%%%%%%%%%%%%%%%%%%%%%%%%%%%%%%%%%%%%%%%%%%%%%%%%

%
%       \begin{problemset}
%               \item[1] Something
%               \item[2] Another
%       \end{problemset}
%
\newenvironment{problemset}{\begin{enumerate}[widest={5.2}]}{\end{enumerate}}


%%%%%%%%%%%%%%%%%%%%%%%%%%%%%%%%%%%%%%%%%%%%%%%%%%%%%%%%%%%%%%%%%%%%%%%%%%%%%%%%
% COMMANDS 
%%%%%%%%%%%%%%%%%%%%%%%%%%%%%%%%%%%%%%%%%%%%%%%%%%%%%%%%%%%%%%%%%%%%%%%%%%%%%%%%

%
% \marginal{...} 
% --------------
% Margin notes with footnote-sized text. 
%
\newcommand{\marginal}[1]{\marginnote{\footnotesize #1 \\}}

%
% \lcolumn{...}, 
% \rcolumn{...}
% ----------------------------
% Columns for use in 'multicol' environment
%
\newcommand{\lcolumn}[1]{\colchunk{#1}}
\newcommand{\rcolumn}[2]{\colchunk{{#1}}\colplacechunks}

%
% \sequence{...}
% --------------
% Sequence in angle brackets <a, b, c, ...>
%
\newcommand{\sequence}[1]{\langle #1 \rangle}

%
% \scr{...}
% ---------
% A shorter way to specify \mathscr{...}
%
\newcommand{\scr}[1]{\mathscr{#1}}

%
% \setcmp{...}
% ------------
% Shortcut for set complement notated 'A \ B'.
%
\newcommand{\setcmp}[2]{#1\setminus{#2}}

%
% \setc{...}
% ----------
% Shortcut for set complement notatetd 'A^c'.
%
\newcommand{\setc}[1]{#1^{\mathsf{c}}}

%
% \vx{...}
% --------
% Vector version of a symbol (bold)
%
\newcommand{\vx}[1]{\mathbf{#1}}


%%%%%%%%%%%%%%%%%%%%%%%%%%%%%%%%%%%%%%%%%%%%%%%%%%%%%%%%%%%%%%%%%%%%%%%%%%%%%%%%
% THEOREMS 
%%%%%%%%%%%%%%%%%%%%%%%%%%%%%%%%%%%%%%%%%%%%%%%%%%%%%%%%%%%%%%%%%%%%%%%%%%%%%%%%

%
%       \newtheoremstyle{stylename}     name of the style to be used
%               {spaceabove}            measure of space to leave above theorem
%               {spacebelow}            measure of space to leave below theorem
%               {bodyfont}              font to use in the body of the theorem
%               {indent}                measure of space to indent
%               {headfont}              font to use in the head of the theorem
%               {headpunctuation}       punctuation between head and body
%               {headspace}             space after theorem head; " " = normal 
%               {headspec}              manually specify head
%
\newtheoremstyle{break}% 
    {}%                    Space above, empty = `usual value'
    {}%                    Space below
    {}%                    Body font (no italic for cheatsheet)
    {}%                    Indent (empty=no indent, \parindent=para indent)
    {\bfseries}%           Thm head font
    {.}%                   Punctuation after thm head
    {\newline}%            Space after thm head: \newline = linebreak
    {}%                    Thm head spec

% Set the default theorem style
\theoremstyle{break}

% Conjecture 
\newtheorem{conjecture}{Conjecture}[section]
\newtheorem*{conjecture*}{Conjecture}

% Postulate 
\newtheorem{postulate}{Postulate}[section]
\newtheorem*{postulate*}{Postulate}

% Principle
\newtheorem{principle}{Principle}[section]
\newtheorem*{principle*}{Principle}

% Axiom 
\newtheorem{axiom}{Axiom}[section]
\newtheorem*{axiom*}{Axiom}

% Definition
\newtheorem*{definition}{Definition}
\newtheorem*{definition*}{Definition}

% Proposition
\newtheorem{proposition}{Proposition}[section]
\newtheorem*{proposition*}{Proposition}

% Claim 
\newtheorem{claim}{Claim}[section]
\newtheorem*{claim*}{Claim}

% Porism 
\newtheorem{porism}{Porism}[section]
\newtheorem*{porism*}{Porism}

% Theorem 
\newtheorem{theorem}{Theorem}[section]
\newtheorem*{theorem*}{Theorem}

% Lemma
\newtheorem{lemma}{Lemma}[section]
\newtheorem*{lemma*}{Lemma}

% Corollary
\newtheorem{corollary}{Corollary}[section]
\newtheorem*{corollary*}{Corollary}

% Intuition 
\newtheorem{intuition}{Intuition}[section]
\newtheorem*{intuition*}{Intuition}

% Consequence 
\newtheorem{consequence}{Consequence}[section]
\newtheorem*{consequence*}{Consequence}

% Remark
\newtheorem*{remark}{Remark}
\newtheorem*{remark*}{Remark}

% Example
\newtheorem{example}{Example}[section]
\newtheorem*{example*}{Example}

% Counterexample 
\newtheorem{counterexample}{Counterexample}[section]
\newtheorem*{counterexample*}{Counterexample}

%%%%%%%%%%%%%%%%%%%%%%%%%%%%%%%%%%%%%%%%%%%%%%%%%%%%%%%%%%%%%%%%%%%%%%%%%%%%%%%%
% SYMBOLS 
%%%%%%%%%%%%%%%%%%%%%%%%%%%%%%%%%%%%%%%%%%%%%%%%%%%%%%%%%%%%%%%%%%%%%%%%%%%%%%%%

%
% These Greek letters are the same as their Latin
% equivalents, so there is no special command in
% LaTeX to indicate their use.
%
% However, for clarity, we have aliased them to their
% Greek names here. 
%
% NOTE:
% \mbox{...} resets its contents to text mode by 
% default, so these letters will not be in italic 
% when you print them in math mode.
%
\newcommand{\Alpha}  {\mbox{A}}
\newcommand{\Beta}   {\mbox{B}}
\newcommand{\Epsilon}{\mbox{E}}
\newcommand{\Zeta}   {\mbox{Z}}
\newcommand{\Eta}    {\mbox{H}}
\newcommand{\Iota}   {\mbox{I}}
\newcommand{\Kappa}  {\mbox{K}}
\newcommand{\Mu}     {\mbox{M}}
\newcommand{\Nu}     {\mbox{N}}
\newcommand{\Omicron}{\mbox{O}}
\newcommand{\omicron}{\mbox{o}}
\newcommand{\Rho}    {\mbox{P}}
\newcommand{\Tau}    {\mbox{T}}
\newcommand{\Chi}    {\mbox{X}}

%
% "Blackboard bold" versions of the most common
% field symbols for the Reals, Integers, etc.
%
\newcommand{\Naturals} {\mathbb{N}}
\newcommand{\Integers} {\mathbb{Z}}
\newcommand{\Reals}    {\mathbb{R}}
\newcommand{\Rationals}{\mathbb{Q}}
\newcommand{\Complexes}{\mathbb{C}}
\newcommand{\Booleans} {\mathbb{B}}
\newcommand{\Field}    {\mathbb{F}}

% Vector space V
\newcommand{\V}    {\mathbf{V}}
% Vector space W 
\newcommand{\W}    {\mathbf{W}}
% Vector 0 
\newcommand{\0}    {\mathbf{0}}

% Vector T 
\newcommand{\T}    {\mathbf{T}}

% Vector N 
\newcommand{\N}    {\mathbf{N}}

% Vector R 
\newcommand{\R}    {\mathbf{R}}


\newenvironment{left-column}{\begin{minipage}[t]{0.4\textwidth}}{\end{minipage}}
\newenvironment{right-column}{\hspace{0.1\textwidth}\begin{minipage}[t]{0.4\textwidth}}{\end{minipage}}



%%%%%%%%%%%%%%%%%%%%%%%%%%%%%%%%%%%%%%%%%%%%%%%%%%%%%%%%%%%%%%%%%%%%%%%%%%%%%%%%
% CONFIGURATION 
%%%%%%%%%%%%%%%%%%%%%%%%%%%%%%%%%%%%%%%%%%%%%%%%%%%%%%%%%%%%%%%%%%%%%%%%%%%%%%%%

% Number equations by section 1.1, 1.2... rather than 1, 2, ...
\numberwithin{equation}{section}

% Reduce bullet size in enumerated lists
\renewcommand{\labelitemi}{$\vcenter{\hbox{\tiny$\bullet$}}$}

% Disable separation between list items
\setlist{nolistsep}

% Margin notes
\setlength{\marginparsep}{1cm}

% Fancy header configuration
\setlength{\headheight}{20pt}
\setlength{\headsep}{30pt}
\pagestyle{fancy}

\definecolor{shadecolor}{RGB}{240,240,240}
\setlength{\FrameSep}{1cm}



%%%%%%%%%%%%%%%%%%%%%%%%%%%%%%%%%%%%%%%%%%%%%%%%%%%%%%%%%%%%%%%%%%%%%%%%%%%%%%%
% HEADER 
%%%%%%%%%%%%%%%%%%%%%%%%%%%%%%%%%%%%%%%%%%%%%%%%%%%%%%%%%%%%%%%%%%%%%%%%%%%%%%%

% Suppress the automatic date
\date{} 

% Generate the title and author
\title{A small context mixing compressor}
\author{Jason Linehan}


%%%%%%%%%%%%%%%%%%%%%%%%%%%%%%%%%%%%%%%%%%%%%%%%%%%%%%%%%%%%%%%%%%%%%%%%%%%%%%%
% DOCUMENT ABSTRACT 
%%%%%%%%%%%%%%%%%%%%%%%%%%%%%%%%%%%%%%%%%%%%%%%%%%%%%%%%%%%%%%%%%%%%%%%%%%%%%%%%
\begin{document}
\maketitle


%%%%%%%%%%%%%%%%%%%%%%%%%%%%%%%%%%%%%%%%%%%%%%%%%%%%%%%%%%%%%%%%%%%%%%%%%%%%%%%%
% DOCUMENT BODY
%%%%%%%%%%%%%%%%%%%%%%%%%%%%%%%%%%%%%%%%%%%%%%%%%%%%%%%%%%%%%%%%%%%%%%%%%%%%%%%%

\section{Explanation}

This is a context-mixing compression algorithm. It is a mathematical machine
made up of mathematical parts. Each part solves a particular problem,
and together, the parts make the algorithm work. The layout and interaction
between these parts is called the architecture of the program. To make 
the architecture easier to see, each part has been broken into its own file.

\subsection{How C programs work}
As a convention, each part, let's call it a {\em module}, is represented by two 
files. Suppose the module's name is {\tt foo}. Then the {\tt foo} module is 
made up of {\tt foo.c}, and {\tt foo.h}. {\tt foo.c} is the C source code 
for the {\tt foo} module. {\tt foo.h} is the header file for {\tt foo.c}.
A header file defines the interface for the module. Symbols in {\tt foo.c} 
(such as the names of functions) are invisible to other parts of the program,
unless we make them available in a header file, and then include that header
file with {\tt \#include "foo.h"} in another part of the program. We say that
the different parts of the program are {\em linked}, using these header files.

You can see other headers being {\tt \#included} at the top of most of 
the files listed here. If the line looks like {\tt \#include <file.h>}, we
are linking a library installed on the system. If it looks like 
{\tt \#include "file.h"}, we are linking a file in our directory.

Except for the three functions in {\tt main.c}, which represent the major
actions of the program, that is: {\tt main()}, {\tt compress()}, and 
{\tt decompress()}, every function is defined in one of these modules.
A module does not have to expose everything in its header file. If there are 
functions or data structures that it uses internally to accomplish its task, 
these can be hidden from the parts of the program that do not need to know 
about them. This is a design rule called the {\em principle of least knowledge}.

\subsection{Getting help}
If you see a function name whose source code does not appear in the listing, 
such as {\tt getc()} on line {\tt 62}, it probably belongs to the C Standard
Library, and is being linked in one of the header files at the top. On any UNIX 
machine, including OS/X, you can type the command {\tt man {\em function}} 
at the terminal, where {\tt {\em function}} is the name of the function, to 
bring up a manual page for that function. You can do the same thing using the 
name of the header files, e.g. {\tt man stdio.h}. These are available because all UNIX 
derivatives are written in C, and these systems are in many ways a programming 
environment for that language. Certain functions, for example {\tt printf()}, 
are available both in the shell and in the C library, so you may need to 
specify that you mean the C library version by typing {\tt 3} after the {\tt man}
command, as in {\tt man 3 printf}. For a listing of all the different manual 
levels, you can (what else?) run {\tt man man}.

\subsection{Tracing the execution}
Every C program begins execution in a function called {\tt main()}. Here, that
function is found in the file {\tt main.c}. If you want to trace the execution
of the program, it goes from the top of {\tt main()} to the bottom of 
{\tt main()}. Simple, right?

\subsection{Programming style conventions}

In programmer-speak, a function which operates on a specific data structure
is called a {\em method} for that data structure. In our world, we specify
the data structure with {\tt struct name\_t}, where the {\tt \_t} means "type".
This is purely a convention. All methods of this data structure will begin with 
{\tt name\_} and the first argument will be a pointer to a data structure of 
that type. For example, the arithmetic coder's data structure is 
{\tt struct ac\_t}, and a method would be {\tt ac\_encode\_bit}.
Except in very special cases, variables in {\tt ALL\_CAPS} represent constant
values, variables in {\tt Title\_case} represent global values relative to a 
given scope, and all other variables are {\tt lowercase}. Tradition from the
old days of teletype is that each line is no more than 80 characters wide.
This turned out to be a good idea in general, since people aren't good at 
scanning left to right. Tabs are 8 spaces wide. There are those who wish it 
were otherwise, but they are wrong to do so.



\section{main.c}
\begin{snugshade}
This file contains the start of the program, {\tt main()}, and the two
primary functions {\tt compress()} and {\tt decompress()}.
{\tt main()} in particular is filled with tedious code to capture and
arrange the arguments given to the program by the user, and plenty of 
tedious filesystem code to resolve paths, create new files, or open old ones.
The messy task of {\tt main()} is to get things set up for the compression or 
decompression. The program could just
as easily be read starting at {\tt compress()} or {\tt decompress()}.
\end{snugshade}

\begin{multicols}{2}
\inputminted[mathescape,
               linenos,
               firstnumber=last,
               numbersep=5pt,
               gobble=0,
               frame=none,
               fontsize=\tiny,
               framesep=2mm]{c}{../main.c}
\end{multicols}

\pagebreak
\section{coder.h}
\begin{snugshade}
This file contains the data structure and methods for the Arithmetic Coder,
probably the single most crucial component of the compression. It is the task
of the Arithmetic Coder to encode a series of predictions (probability values)
about whether the next bit will be a 0 or a 1.
\end{snugshade}

\inputminted[mathescape,
               linenos,
               firstnumber=last,
               numbersep=5pt,
               gobble=0,
               frame=none,
               fontsize=\footnotesize,
               framesep=2mm]{c}{../coder.h}
\section{coder.c}
\begin{multicols}{2}
\inputminted[mathescape,
               linenos,
               firstnumber=last,
               numbersep=5pt,
               gobble=0,
               frame=none,
               fontsize=\tiny,
               framesep=2mm]{c}{../coder.c}
\end{multicols}


\pagebreak
\section{mixer.h}
\begin{snugshade}
This file contains the data structure and methods for the neural network which
mixes the predictions of different context models. A neural network is a means
to minimize an error function by optimizing a set of weights.

An extremely brief glance at what a neural network is and does:

A neural network is a special kind of function. Given n binary inputs, the 
neural network will give 1 binary output. Each of the n inputs is associated
with one of n weights. The inputs, multiplied by their weights, are summed,
and if the result of the sum is above a certain threshold, the output is 1.
If it is below that threshold, the output is 0. In the real world, we need
to be able to take the derivative of the output, so an activation function
$K$ is chosen, typically a sigmoid function such as the logistic function,
rather than using a discrete threshold.
\begin{align*}
        f(x_0,x_1,...,x_n) = K(\sum_{i=0}^n w_ix_i)
\end{align*}

The major feature of the neural network is that the weights $w_i$ are adaptive,
that is, the algorithm can alter the weights to change the function, in a 
process called learning. 

Consider the output of the network as a prediction of the "right" answer,
given the inputs $x_0,...,x_n$. If we obtain the actual answer, we can compare
it with the output of the neural network, and compute some error function,
typically the mean squared error (MSE).

Then, the weights are updated, in order to minimize this error function.
The method used to update the weights is an algorithm called backpropagation,
which itself uses another technique called gradient descent.
\end{snugshade}
\inputminted[mathescape,
               linenos,
               firstnumber=last,
               numbersep=5pt,
               gobble=0,
               frame=none,
               fontsize=\small,
               framesep=2mm]{c}{../mixer.h}
\section{mixer.c}
\begin{multicols}{2}
\inputminted[mathescape,
               linenos,
               firstnumber=last,
               numbersep=5pt,
               gobble=0,
               frame=none,
               fontsize=\tiny,
               framesep=2mm]{c}{../mixer.c}
\end{multicols}

\pagebreak
\section{predictor.h}
\begin{snugshade}
This file contains the data structure and methods for the {\tt SMOOTH()} 
function, which performs something called Secondary Symbol Estimation (SSE),
which is an unpublished folkloristic procedure for refining a probability
value.
\end{snugshade}
\inputminted[mathescape,
               linenos,
               firstnumber=last,
               numbersep=5pt,
               gobble=0,
               frame=none,
               fontsize=\small,
               framesep=2mm]{c}{../predictor.h}
\section{predictor.c}
\begin{multicols}{2}
\inputminted[mathescape,
               linenos,
               firstnumber=last,
               numbersep=5pt,
               gobble=0,
               frame=none,
               fontsize=\tiny,
               framesep=2mm]{c}{../predictor.c}
\end{multicols}

\pagebreak
\section{model.h}
\begin{snugshade}
This file contains the code for the {\tt MODEL()} function, which computes a
probability based on recent bits. This is the statistical model of the input
stream.

The main action of the model is contained in the {\tt MODEL()} function, which
controls how the various parts of the history and contexts are arranged to make
predictions, and how those predictions are fed to the neural network. 

The name of the function, {\tt MODEL}, illustrates an unfortunate situation
familiar to programmers, as "model" is both a noun and a verb. So, is this
function the model, or does this function do the modelling? The answer is
that it controls the modelling, and so to other layers of the program, for
all intents and purposes, it "is" the model, as information from the input
stream goes in, and a prediction comes out.

However the actual models are data structures, and in fact several different
kinds of models can be combined and mixed by the {\tt MODEL} function to get 
different results. For our purposes, only one kind of model is used, the 
simplest kind, a simple stationary model ({\tt ssm\_t}).

The operation of the SSM model is simple. It contains a large table of 
probabilities, all initially set to 1/2. When a byte is being processed
by the model, the previously-seen byte is considered to be the "current context",
i.e. the context in which predictions are now being made about the incoming
bits of the current byte. The previous byte, the "context", is used to retreive
a probability in the table. 

Since the context is just an integer value, call the value 'b'. We simply 
select the b-th item in the table, and that item will represent the probability 
of the next bit being a 1, if byte b is the current context. 

When we see a new bit, we modify the probability stored in the b-th item
of the table according to an update rule. It is this update rule that 
determines how the model behaves statistically.
Since this is a stationary model, we simply increment or decrement the
probability according to what we see.

Once we are done with the current byte, that byte becomes the new context,
and a new probability value is looked up in the table, and the process
continues.
\end{snugshade}
\inputminted[mathescape,
               linenos,
               firstnumber=last,
               numbersep=5pt,
               gobble=0,
               frame=none,
               fontsize=\small,
               framesep=2mm]{c}{../model.h}
\section{model.c}
\begin{multicols}{2}
\inputminted[mathescape,
               linenos,
               firstnumber=last,
               numbersep=5pt,
               gobble=0,
               frame=none,
               fontsize=\tiny,
               framesep=2mm]{c}{../model.c}
\end{multicols}

\pagebreak
\section{util.h}
\begin{snugshade}
Here we have a grab-bag of different items of widely varying importance.
\end{snugshade}
\inputminted[mathescape,
               linenos,
               firstnumber=last,
               numbersep=5pt,
               gobble=0,
               frame=none,
               fontsize=\small,
               framesep=2mm]{c}{../util.h}
\section{util.c}
\begin{multicols}{2}
\inputminted[mathescape,
               linenos,
               firstnumber=last,
               numbersep=5pt,
               gobble=0,
               frame=none,
               fontsize=\tiny,
               framesep=2mm]{c}{../util.c}
\end{multicols}


%\begin{minted}[mathescape,
               %linenos,
               %numbersep=5pt,
               %gobble=2,
               %frame=lines,
               %escapeinside=$,
               %framesep=2mm]{c}
               %$\input{../main.c}$
%\end{minted}






\end{document}
